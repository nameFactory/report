\documentclass[12pt,a4paper]{article}
\usepackage[T1]{fontenc}
\usepackage[stretch=20]{microtype}
\usepackage[utf8]{inputenc}
\usepackage{mathptmx}
\linespread{1.3}
\usepackage{indentfirst}
\usepackage{secdot}
\sectiondot{subsection}
\usepackage{listings}
\usepackage{amsmath}
\usepackage{graphicx}
\graphicspath{ {screenshots/} }

\newcommand{\code}[1]{\texttt{#1}}
\newcommand{\norm}[1]{\left\lVert#1\right\rVert}

\renewcommand*{\figurename}{Ryc.}

\begin{document}
\begin{table}[t]
\centering
\begin{tabular}[t]{lcr}
& POLITECHNIKA WARSZAWSKA & \\
& WYDZIAŁ MATEMATYKI & \\
& I NAUK INFORMACYJNYCH &
\end{tabular}
\end{table}

\author{Piotr Filarski \\ Marta Marciszewicz \\ Adrian Sadłocha}
\title{Aplikacje mobilne: Android \\ NameFactory -- raport}
\date{Warszawa, 14 czerwca 2017}

\maketitle

\section*{Opis aplikacji}
Wybór imienia dla dziecka jest istotną decyzją, zarówno w życiu rodziców, jak i samego dziecka.
Ponadto, imiona trudno porównywać między sobą, ze względu na ich liczbę.
Nasza aplikacja służy do budowania \textit{rankingu imion} na podstawie pojedynczych wyborów między parami imion.
Ponadto udostępniamy użytkownikom \textit{ranking globalny} oraz \textit{opisy imion}.
\newpage

\section{Działanie aplikacji}
\subsection{Tworzenie rankingu}

Aplikacja ma na celu tworzenie rankingów użytkownika, dlatego też pierwszym krokiem po odpaleniu aplikacji, powinno być stworzenie rankingu. W tym celu należy nacisnąć przycisk w prawym dolnym rogu ekranu. Przeprowadzi nas on do okna wyboru filtrów dla nowo tworzonego rankingu. W bierzącej wersji aplikacji można jedynie wybrać kraj pochodzenia imienia oraz płeć związaną z imieniem. 
Po skonfigurowaniu rankingu ponownie wybieramy przycisk w prawym dolnym rogu, co spowoduje utworzenie nowego rankingu i powrót do głównego okna aplikacji. Rankingi tworzone są z domyślną nazwą, którą można później zmienić wchodząc w dany ranking i wybierając opcję 'Change ranking name' z prawego górnego rogu ekranu. 
Po stworzeniu nowego rankingu można się zabrać za ocenianie imion.

\subsection{Ocenianie imion}

Podstawowym celem aplikacji jest zbudowanie spersonalizowanego rankingu imion.
W tym celu, użytkownik tworzy ranking, wybierając przy tym płeć dziecka.
Następnie użytkownikowi prezentowane są pary imion, z których użytkownik wybiera to, które mu się bardziej podoba.
Na podstawie takich pojedynczych spotkań, budowany jest spersonalizowany ranking imion danej płci.

\begin{figure}[h]
    \caption{Główny ekran po włączeniu -- lista rankingów użytkownika}
    \centering
    \includegraphics[width=0.5\textwidth]{rankings}
\end{figure}

Ze względu na mnogość imion, chcemy, aby użytkownik częściej oceniał imiona, które mu się podobają.
Dlatego -- w przypadku łączności z internetem -- aplikacja pobiera listę par do ocenienia (dla danego rankingu) z serwera.
Na serwerze jest zaimplementowany sprytniejszy niż losowy algorytm wybierania par do oceny.
W przypadku braku łączności z internetem, klient używa naiwnego (w tym przypadku: losowego) algorytmu doboru par.

\begin{figure}[h]
    \caption{Ekran oceniania pary imion}
    \centering
    \includegraphics[width=0.5\textwidth]{match}
\end{figure}

\subsection{Ranking globalny}
Każdy wybór użytkownika jest wysyłany na nasz serwer, w celu stworzenia globalnego rankingu.
W przypadku braku połączenia z internetem, wysłanie tychże danych nastąpi po jego uzyskaniu.
Do tego czasu poszczególne oceny imion będą trzymane w kolejce.

\begin{figure}[h]
    \caption{Widok rankingu globalnego}
    \centering
    \includegraphics[width=0.5\textwidth]{top}
\end{figure}

\subsection{Opisy imion}
Niektórzy użytkownicy przywiązują dużą wagę do znaczenia imienia.
Dlatego udostępniamy możliwość przeczytania opisu wybranych imion.

\begin{figure}[h]
    \caption{Widok opisu imienia}
    \centering
    \includegraphics[width=0.5\textwidth]{name_description}
\end{figure}

\section{Opis rozwiązania}
\subsection{Wstęp}

W projekcie można wyróżnić dwie zasadnicze części: klient (jako aplikacja na system Android) oraz serwer (wystawiający API -- poprzez HTTPS -- skrypt pythonowy).
Jako systemu kontroli wersji użyliśmy narzędzia Git.
Koordynację działań zespołu wspomagał GitHub, na którym umieściliśmy 3 repozytoria: dla kodu klienta, serwera oraz tekstu raportu.

\subsection{Klient}
Naturalnym wyborem do stworzenia klienta był język Java.
Cały zespół wykorzystywał Android Studio jako główne IDE.
Do testów użyty został framework JUnit.

Ze względu na komunikację klienta z serwerem, przydatne okazały się biblioteki Retrofit oraz gson.
W zestawieniu umożliwiają one wygodną realizację mapowania odpowiednich \textit{restfulowych} odpowiedzi na \textit{javowe} obiekty (i na odwrót).

Poza powyższymi bibliotekami, nie mieliśmy potrzeby wykorzystywania innych niż te, które są dostarczane przez domyślne androidowe SDK.

Klient używa bazy danych SQLite do przechowywania informacji dot. utworzonych rankingów, loginu i hasła użytkownika, a także bazy imion.

\subsection{Serwer}
Część serwerowa została napisana przy użyciu języka Python 3 oraz frameworku Flask, który służy do tworzenia aplikacji webowych.
API umożliwia następujące akcje:

\begin{itemize}
    \item Utworzenie konta użytkownika;
    \item Utworzenie rankingu;
    \item Dodanie oceny danej pary imion;
    \item Pobranie listy par do ocenienia;
    \item Pobranie globalnego rankingu;
    \item Pobranie aktualnej bazy imion (wraz z opisami).
\end{itemize}

API jest udostępnione poprzez HTTPS.
Obsługę zapytań (oraz certyfikatów TLS/SSL) zapewnia nginx, który w naszym projekcie pełni funkcję WWW oraz serwera proxy.
API do wymiany danych wykorzystuje format JSON.

Serwer wykorzystuje również bibliotekę SQLAlchemy, która jest popularnym pythonowym ORM-em.
Umożliwa to stosunkową łatwą zmianę wykorzystywanej bazy danych w razie takiej potrzeby.
Do celów deweloperskich wykorzystywaliśmy bazę SQLite.

API jest dostępne pod adresem https://api.namefactory.pl.

\subsection{Ranking}
W celu stworzenia spersonalizowanego rankingu imion, wykorzystaliśmy tzw. ranking szachowy (elo).
Używany jest on zarówno przez klienta -- do budowania rankingów danego użytkownika -- jak i przez serwer -- do budowania rankingu globalnego.
Aktualnie używany jest w tej samej wersji, ale są to niezależne implementacje, umożliwiające -- przykładowo -- zmianę sposobu liczenia rankingu globalnego bez wpływania na rankingi użytkowników.

Zastosowaliśmy ranking szachowy z domyślną wartością dla każdego imienia wynoszącą \(1200\), oraz ze współczynnikiem \(K\) równym \(64\).
Wysoki współczynnik \(K\) umożliwia szybkie zmiany w rankingu, kosztem jego stabilności.
Uznaliśmy to za stosowny kompromis, ponieważ imion jest wiele, a nie chcemy wymagać od użytkownika dużej liczby ocen, aby ranking przybrał postać bliską prawdy.

\end{document}

